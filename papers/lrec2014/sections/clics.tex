CLiCs (\emph{\textbf{C}ross-\textbf{Li}nguistic \textbf{C}olexification\textbf{s}},
\url{http://lingulist.de/clics/}) is an online
database of synchronic lexical associations (``polysemies", but more precisely:
\emph{colexifications}, see below) for 1288 concepts translated into currently 215 language varieties of the world. 
Large databases offering lexical
information on the world's languages are already readily available for research in different online
sources. However, the information on tendencies of meaning associations available in these databases
is not easily extractable from the sources themselves. This is why CLiCs was created.  
It is designed
to serve as a data source for work in lexical typology, diachronic semantics, and research in
cognitive science that focuses on natural language semantics from the viewpoint of cross-linguistic
diversity. Furthermore, CLiCs can be used to assess the plausibility of semantic
connections between possible cognates in the establishment of genetic relations between languages.
Table \ref{tab:clics} gives an example on the basic structure of the data in CLiCs. 
 
\begin{table}[b]
    \centering
\resizebox{8cm}{!}{%
\tabular{|l|l|l|l|}
\hline
\bf Concept &\bf IDS-Key &\bf Families &\bf Languages \\\hline
\hline
\rowcolor{lightgray} money &11.43 &15 &33 \\\hline
\rowcolor{lightgray} coin &11.44 &9 &13 \\\hline
\rowcolor{lightgray} iron &9.67 &3 &3 \\\hline
\rowcolor{lightgray} gold &9.64 &2 &2 \\\hline
\rowcolor{lightgray} tin, tinplate &9.69 &2 &2 \\\hline
\rowcolor{lightgray} white &15.64 &2 &2 \\\hline
blunt, dull &15.79 &1 &1 \\\hline
bright &15.57 &1 &1 \\\hline
chest &4.4 &1 &1 \\\hline
clock, timepiece &14.53 &1 &1 \\\hline
copper, bronze &9.66 &1 &1 \\\hline
earring &6.77 &1 &1 \\\hline
hammer &9.49 &1 &1 \\\hline
helmet &20.33 &1 &1 \\\hline
jewel &6.72 &1 &1 \\\hline
lead (noun) &9.68 &1 &1 \\\hline
price &11.87 &1 &1 \\\hline
razor &6.93 &1 &1 \\\hline
saw &9.48 &1 &1 \\\hline\endtabular}
\caption{Common colexifications involving the concept ``silver" in CLiCs. Concepts which are
expressed by the same word form in more than two different language families are shaded gray.}
\label{tab:clics}
\end{table}

\subsection{Homonymy, polysemy, and colexification}
%CLiCs can help to identify regularities in semantic
%change and conceptual associations. Innovative semantic change consists in the acquisition of a new
%meaning $n$ by a given word $w$, which has one or more original meanings $o_{i}, i \in \mathbb{N}$.
%The new meaning $n$ is associated in some way -- for example metaphorically or metonymically -- to
%$o_k$, one of the original meanings of $w$ \cite{blank1997}. As a consequence, $w$ carries
%-- at least for a certain time -- both meanings $n$ and $o$, between which there is a conceptual
%relation. This is referred to as \emph{polysemy (of $w$)}. Conversely, historical linguists
%conclude abductively from the polysemy of $w$ that one of the meanings of $w$ is likely to have
%emerged on the basis of the other. Therefore, polysemy can be used to reconstruct and analyze
%semantic change \cite{traugottdasher2002}.
 
From the analytical perspective, polysemy has
to be distinguished from homonymy and semantic vagueness. \emph{Homonymy} refers to the
``accidental'' verbalization of at least two meanings by the same sound chain, without a conceptual
relation between $n$ and $o$ that is more than coincidental. \emph{Contextual variation}
designates the adaption of a lexicalized meaning to contextual factors in an utterance. Although
historical and synchronic criteria have been proposed to distinguish polysemy from homonymy, and
contextual variation can be tested by resorting to categorization \cite{blank1997}, the
differentiation depends on the individual analysis of every single word and is not entirely
objective. Therefore, it is difficult for quantitative investigations to provide this differentiation
in advance.

In the context of CliCs, we use the term \emph{colexification} (coined to our knowledge by Fran\c{c}ois
2008) \nocite{francois2008}
to refer to the situation when two or more of the meanings in our lexical sources are covered in a
language by the same lexical item. For instance, we would say that Russian \emph{рука} colexifies ``hand"
and ``arm", that is, concepts that are semantically related to each other. Roughly speaking,
colexification can correspond either to polysemy or contextual variation in lexical semantic analyses.
Since CLiCs is not based on such analyses that would allow us to further discriminate between the
two, we chose colexification as a label that deliberately does not make a commitment with regard to
this distinction. However, as we will show below, quantitative approaches are available to rule out effects of accidental homonymy. 
%To take account of the difficulty of distinguishing neatly between cases of contextual variation,
%polysemy, and homonymy, we stick to the term \emph{colexification} \cite{francois2008} as referring
%to any situation in which several meanings are \emph{colexified} by the same sound chain. 
%This
%terminological convention has additional advantages: 1) Whereas polysemy describes the property of a
%word, colexification describes the property of a group of concepts and thus fits in much better with
%the onomasiological approach of CLiCs. 2) We avoid unintended implications based on the
%structuralist origin of the term \emph{polysemy} \cite{breal1897} and readily reflect the fact that
%not all synchronic lexical associations in CLiCs are the result of historical change processes.

\subsection{Data and sources of CLiCs}
CLiCs offers information on colexification in 215 different language varieties
covering 50 different language families. All language varieties in our sample comprise a total of
290,760 words covering 1,288 different concepts. Using a strictly automatic procedure, we identified
45,282 cases of colexification that correspond to 16,043 different links between the 1,288 concepts
covered by our data. 

Currently CLiCs utilizes three different sources, all of which are freely available online
themselves.  (1) The \emph{Intercontinental Dictionary Series} (IDS, Key and Comrie 2007) 
\nocite{Key2007} features lexical
data for 233 world languages. IDS data were provided mostly by experts on the respective languages,
although in some cases published written sources have been used. There are 1,310 entries to be
filled for each language, though, of course, there are gaps in coverage for individual languages.
The list of concepts is inspired by Buck \shortcite{Buck1949}. Of all 233 languages in the IDS, 178 were
automatically cleaned and included in CLiCs.\footnote{In all cases, we ignored proto-languages and
archaic languages (like Latin and Old Greek), and those languages which
did not have enough coverage in terms of lexical items.} (2) The IDS list, in turn, provides the basis for the
choice of meanings in the \emph{World Loanword Database} (WOLD, Haspelmath and Tadmor 2009).
\nocite{Wold2009} The principal aim
of this source is to provide a basis for generalizations on the borrowability of items in different
parts of the lexicon. The WOLD data consist of vocabularies of between 1,000-2,000 items for 41
languages, with annotations about the borrowing history of particular items where applicable. WOLD
data was coded by experts on the respective languages, in some cases also with the aid of extant
sources. Of all 41 languages in WOLD, 33 languages, which were not yet present in the IDS, 
are included in CLiCs.  (3) The \emph{Logos Dictionary}
(\url{http://www.logosdictionary.org}) is a freely accessible multilingual online dictionary that is
regularly updated online by a network of professional translators. It offers lexical data for more
than 60 different languages. We manually extracted lexical data for 4 languages that were neither
present in IDS nor in WOLD. 


\subsection{Network modeling of CLiCs}
As mentioned above, there is no guarantee that lexical associations within CLiCs are due to
historical reasons or due to chance. For example, there are three attested links between the
concepts ``arm" and ``poor" in the current version of CLiCs, which are due to homonymy in some
Germanic languages (German, Dutch, and Yiddish).
 
In order to distinguish strong
association tendencies from
spuriously occurring associations and to rule out cases of accidental homonymy, \newcite{List2013a} model cross-linguistic colexification data as a weighted network
in which nodes represent concepts and weighted edges between the nodes represent the number of
attested colexifications in the data.  With the help of \emph{community detection analyses}, strongly
interconnected regions in the colexification network can be identified. \newcite{List2013a} apply
a weighted version of the community detection algorithm by \newcite{Girvan2002} to a
cross-linguistic colexification network consisting of 1252 concepts translated into 195 languages
covering 44 language families. Their analysis yielded a total of 337 communities, with 104 communities
consisting of 5 and more nodes and covering 68\% of all concepts. A qualitative survey of the
largest communities showed that most of them constitute meaningful units, and accidental homologies
were successfully excluded.
\subsection{Limitations and Caveats} \label{caveats}
The structure of the data in CLiCs is a direct image of the structure of the data in IDS and WOLD
and does not involve a reanalysis of any sort on our behalf. However, it must be emphasized that the
meaning associations reported in CLiCs are recovered from sheer identity of form in different cells
in the sources we have used, and do not necessarily rest on language-internal semantic analysis.
Furthermore, we have no control over artifacts that (a) may have arisen in the process of data
gathering themselves, (b) were created by mapping the predefined concepts onto the actual languages,
and (c) were introduced when cleaning parts of the data automatically in which the textual coding
was not provided in a consistent way.
 
A further problem that may arise when using CLiCs is that the coverage of the world's languages in
both IDS and WOLD is biased towards certain regions of the world. In the case of IDS, South American
languages and languages of the Caucasus are overrepresented. In the case of WOLD, languages of
Europe figure particularly prominently. Since it is possible and even expectable that certain
polysemies in the lexicon are frequent or even restricted to certain areas of the world, it is
important to take appropriate measures to rule out unwarranted generalizations due to areal
effects.

