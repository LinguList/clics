

CLiCs (\emph{\textbf{C}ross-\textbf{Li}nguistic \textbf{C}olexification\textbf{s}},
\url{http://clics.lingpy.org}) is an online database of synchronic lexico-semantic associations
in 221 languages and language varieties of the world. CLiCs exploits already existing large online
lexical databases, but has the advantage that it makes visible the relationships between meanings
and forms in the object languages, something which is not easily possible using the interfaces of its
sources themselves. Table \ref{tab:clics} gives an example on the basic structure of the data in CLiCs.
 
\begin{table}[t]
    \centering
    \href{\clics{all.php?gloss=silver}}{%
\resizebox{8cm}{!}{%
\tabular{|l|r|r|r|}
\hline
\bf Concept &\bf IDS-Key &\bf Families &\bf Languages \\\hline
\hline
\rowcolor{lightgray} money &11.43 &15 &33 \\\hline
\rowcolor{lightgray} coin &11.44 &9 &13 \\\hline
\rowcolor{lightgray} iron &9.67 &3 &3 \\\hline
\rowcolor{lightgray} gold &9.64 &2 &2 \\\hline
\rowcolor{lightgray} tin, tinplate &9.69 &2 &2 \\\hline
\rowcolor{lightgray} white &15.64 &2 &2 \\\hline
blunt, dull &15.79 &1 &1 \\\hline
bright &15.57 &1 &1 \\\hline
chest &4.40 &1 &1 \\\hline
clock, timepiece &14.53 &1 &1 \\\hline
copper, bronze &9.66 &1 &1 \\\hline
earring &6.77 &1 &1 \\\hline
hammer &9.49 &1 &1 \\\hline
helmet &20.33 &1 &1 \\\hline
jewel &6.72 &1 &1 \\\hline
lead (noun) &9.68 &1 &1 \\\hline
price &11.87 &1 &1 \\\hline
razor &6.93 &1 &1 \\\hline
saw &9.48 &1 &1 \\\hline\endtabular}}
\caption{Common colexifications involving the concept ``silver" in CLiCs. Concepts which are
expressed by the same word form in more than one language family are shaded gray. In order to browse
the table on the CLiCs website, use the following URL: \url{\clics{all.php?gloss=silver}}.}
\label{tab:clics}
\end{table}

\subsection{Homonymy, polysemy, and colexification}
A well-known concept from lexical semantic analysis is that of \emph{polysemy}. It refers to the situation
in which a lexical item possesses more than one identifiable sense between which there is a
conceptual relation. A number of tests are available to distinguish polysemy from \emph{semantic
vagueness}, in which a division into distinct senses is not warranted. From an analytical perspective
polysemy has to be further distinguished from {homonymy} and contextual variation.
\emph{Homonymy} refers to
the "accidental" verbalization of at least two meanings by the same sound chain, without any
conceptual relation that is more than coincidental. \emph{Contextual variation}
designates the adaption of a lexicalized meaning to contextual factors in an utterance. Although
historical and synchronic criteria have been proposed to distinguish polysemy from homonymy, and
contextual variation can be tested by resorting to categorization \cite{blank1997}, the
differentiation depends on the individual analysis of every single word and is not entirely
objective. Hence, it is difficult for quantitative investigations to provide this differentiation
in advance.
Here, we use the term \textit{colexification} (originally from \newcite{francois2008})%\nocite{francois2008}
to refer to the situation
in which two or more of the meanings in our sources correspond to the same lexical item in one of
the languages. For instance, we would say that Wayuu [guc] colexifies ``good'' and ``beautiful'' by means of the word form \textit{anas\textbari}.\footnote{See also the example of `money' and `silver' in the case study in Section \ref{case study} below.}
Colexification is thus a deliberately ambiguous label that allows us to avoid making a commitment in
each case as to the adequate lexical semantic analysis.
Roughly speaking,
colexification can correspond either to polysemy or contextual variation in lexical semantic analyses.
Since CLiCs is not based on such analyses that would allow us to further discriminate between the
two, we chose colexification as a label that deliberately does not make a commitment with regard to
this distinction. However, as we will show below, quantitative approaches are available to rule out effects of accidental homonymy. 

\subsection{Data and sources of CLiCs}
CLiCs (Version 1.0) offers information on colexification in 221 different language varieties
covering 64 different language families.\footnote{This count includes 12 language isolates, and 3
unclassified languages, according to the classification schema of Ethnologue \cite{Lewis2013}.} All language
varieties in our sample comprise a total of 301~498 words covering 1~280 different
concepts.\footnote{Since some concepts are expressed by more than one word in the respective
languages, the number of words is higher than the expected one (282~880) if multiple synonyms per
concept were not allowed.} Using a strictly automatic procedure, we identified 45~667 cases of
colexification that correspond to 16~239 different links between the 1~280 concepts covered by our
data. 

At present, four sources feed into CLiCs: (1) The \emph{Intercontinental Dictionary Series} (IDS,
\newcite{Key2007}), offering lexical data for 233 languages and language varieties of the world.
Ideally, datasets for each language contain 1,310 entries, though coverage differs in completeness
for individual languages. Of all 233 languages in IDS, 178 were automatically cleaned and included
in CLiCs. (2) The \emph{World Loanword Database} (WOLD, \newcite{Wold2009}), the main goal
of which has to do with identifying lexical borrowings, but which nevertheless also provides general
lexical data for 41 languages. The vocabularies for the individual languages differ somewhat in
their size, ranging somewhere between 1,000 and 2,000 items. 33 of the 41 vocabularies are included
in CLiCs.  (3) Data for four languages neither represented in IDS nor WOLD were added from the
\emph{LOGOS} dictionary (\url{http://www.logosdictionary.org}), a multilingual online dictionary.
(4) Additional data for six Himalayan languages was taken from the \emph{Språkbanken} project
(University of Gothenburg, \url{http://spraakbanken.gu.se}).\footnote{In all cases, we ignored
proto-languages and archaic languages (like Latin and Old Greek), and those languages which did not
have enough coverage in terms of lexical items.}

%\nocite{Wold2009}
\subsection{Network modeling of CLiCs}
As mentioned above, there is no guarantee that lexical associations within CLiCs reflect conceptual
associations. For example, there are three attested links between the concepts ``arm" and ``poor" in
the current version of CLiCs, which are due to homonymy in some Germanic languages (German, Dutch,
and Yiddish).
 
In order to distinguish strong association tendencies from spuriously occurring associations and to
rule out cases of accidental homonymy, \newcite{List2013a} model cross-linguistic colexification
data as a weighted network in which nodes represent concepts and weighted edges between the nodes
represent the number of attested colexifications in the data.  With the help of \emph{community
detection analyses}, strongly interconnected regions in the colexification network can be
identified. 
Communities are groups of nodes in a network
`within which the connections are dense but between which they are sparser' \cite{Newman2004}.
\newcite{List2013a} apply a weighted version of the community detection algorithm by
\newcite{Girvan2002} to a cross-linguistic colexification network consisting of 1,252 concepts
translated into 195 languages covering 44 language families. Their analysis yielded a total of 337
communities, with 104 communities consisting of 5 and more nodes and covering 68\% of all concepts.
A qualitative survey of the largest communities showed that most of them constitute meaningful
units, and accidental homologies were successfully excluded.

\subsection{Limitations and Caveats} \label{caveats}

The data structure in CLiCs directly mirrors the data structure of the sources we used. We did not manipulate or
reanalyze the data in any way, to the effect that the reliability of CLiCs is greatly dependent on
the reliability of its sources. Additionally, it should be pointed out that we also cannot rule out the
possibility of artifacts arising from automatic data cleaning in cases where textual coding of the
data was inconsistent. As for its actual application, it also must be borne in mind that CLiCs
reflects a certain bias regarding the geographical locations of the languages included in its
sources: IDS features many languages of South America and the Caucasus, while WOLD includes a
disproportionate percentage of languages of Europe. Hence, the sheer frequency of instances of a
particular colexification pattern in CLiCs may be misleading insofar as a pattern is very robust
cross-linguistically, but actually is so only in certain regions of the world. We have not
implemented any computational method in CLiCs to balance out the picture \emph{a posteriori}. Since
we nevertheless want to present potential users of CLiCs with the possibility to assess possible
areal patterns in the data, we include a powerful visualization that enables them to detect areal
imbalances in colexification patterns in individual cases themselves.


